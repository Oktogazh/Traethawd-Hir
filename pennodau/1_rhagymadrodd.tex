\chapter{Cyflwyniad}    
    \abbrv{IIA}{Iaith (neu Ieithoedd) Isadnodd}
Mae'r bennod gyntaf hon yn cyflwyno'r cyd-destun, cwmpas a phwrpas y traethawd hir hwn. Yn benodol, daw'r drydedd adran â'r rôl y mae gan dechnolegau addysgol i'w chwarae i'r amlwg, naill ai wrth gefnogi neu beryglu ymhellach ieithoedd isadnodd (IIA), yn dibynnu ar ai yw'r dechnoleg yn cael ei chynllunio i ddysgu ieithoedd sydd eisoes yn bygwth ieithoedd eraill. Gellir darllen yr adran hon fel cyflwyniad cyffredinol i faes technolegau addysgol i'r rhai hynny sy'n pryderu ynghylch tynged ieithoedd isadnodd, neu fel cyflwyniad i bryderon ieithoedd isadnodd i'r rhai hynny sy'n gweithio ym maes y technolegau addysgol.

\section{Strwythur y Traethawd}
Mae'r traethawd hwn yn cyflwyno Leksis, y sydd yn brawf geirfa adnabyddiadol newydd wedi ei deilwra ar gyfer ieithoedd isadnodd. Mae'r bennod gyntaf yn egluro'r rhesymeg tu ôl i brawf o'r fath. Daw'r ail bennod â rhannau o'r llenyddiaeth sydd ar gael o wahanol feysydd at ei gilydd, gan amrywio o ieithyddiaeth gymhwysol i ddamcaniaeth gwybodaeth (information theory), er mwyn gosod y sylfeini ar gyfer profion geirfa graddadwy sydd yn addasedig i gyfyngiadau a chyd-destun yr ieithoedd isadnodd. Mae'r drydedd bennod yn cyflwyno cynllun prawf cychwynnol ar gyfer y Llydaweg. Y bedwaredd bennod y sydd yn dadansoddi canlyniadau'r brawf i asesu perthnasedd y dewisiadau dylunio. Yn olaf, mae'r bumed bennod yn asesu gwerth a chyfyngiadau'r brawf, yn ogystal â chyflwyno barn wybodus ar yr anghenion penodol i ieithoedd isadnodd mewn perthynas â thechnolegau addysgol ac ieithyddol.

\section{Nod, Amcanion a Chwestiwn Ymchwil}
Mae ieithoedd isadnodd yn wynebu heriau arbennig mewn byd lle mae gwyddor data wedi gwneud maint yn frenin pob rhinwedd. Prif nod y gwaith hwn yw optimeiddio dysgu ieithoedd isadnodd. Gan mai problem hanfodol unrhyw broses optimeiddio yw'r metrig y sydd rhywun yn anelu at ei optimeiddio, mae hyn yn arwain at ddatblygu profion iaith cyflym a lleiafol a gyflwynir yma. Yn benodol, yr amcan yw dod o hyd i ffyrdd i wneud iawn am broblem prinder adnoddau trwy ddatblygu dulliau a thechnegau sydd wedi eu cynllunio i weithio yn y cyd-destun prinder hwn yn gyntaf, yn hytrach na chario trosodd i ieithoedd isadnodd ddulliau a thechnegau sy'n rhy ddwys o ran data. Am y rhesymau hyn, y cynigir y cwestiwn ymchwil canlynol:

\textit{Ai ellir mesur cynnydd hyfedredd iaith yn ddibynadwy mewn ieithoedd isadnodd?}

Wrth gwrs, ni all cyfyngiadau amser y traethawd hwn ganiatáu ar gyfer astudiaeth raddfa fawr o gynnydd mewn hyfedredd iaith grwpiau cyfan o ddysgwyr dros gyfnod llawn cwrs. Ond trwy astudio'r llenyddiaeth yn drwyadl a chyflwyno gwerthusiad cynnar, y bwriadir gallu cynnig dadl gadarn erbyn diwedd y gwaith hwn.

\section{Cefndir a Chymhelliad}
    \subsection{Terminoleg: DAmA ac EdTech}
        \abbrv{EdTech}{Technolegau Addysg}
        \abbrv{DAmA}{Deallusrwydd Artiffisial mewn Addysg (AIED, AIEd)}
Mae ymchwil academaidd fodern ar dechnolegau addysgol yn bennaf yn disgyn dan yr ymbarél ``Deallusrwydd Artiffisial mewn Addysg'' (DAmA neu AIED am ``AI in Education''). Mae'r derminoleg hon yn dominyddu'r maes oherwydd ``Cymdeithas y DAmA Ryngwladol'' (International AIED Society), a sefydlwyd ym 1993, ac effaith strwythurol rhifynnau ei chyfnodolyn a'i chynadleddau. Fodd bynnag, gellir defnyddio y DAmA weithiau'n eithaf cyfnewidiol gydag EdTech, sef ``Technolegau Addysgol'', sy'n derminoleg fwy cyfeiriol at gynnyrch a'r farchnad, term sy'n perthyn mwy i neologismau eraill fel ``FinTech'', ``BioTech'' ac yn y blaen. Gellir ystyried cwmnïau addysgol fel Duolingo neu Rocket Language fel cwmnïau EdTech mewn cyd-destyn diwydiannol, ond fel rhan o faes y DAmA pan siaradir gan ymchwilwyr. Mewn fformiwleiddiad arall, EdTech yw'r DAmA gyda model busnes.

    \subsection{Ieithoedd Isadnodd mewn Technolegau Addysgol}
        \abbrv{PIN}{Prosesu Iaith Naturiol}
        \abbrv{WEIRD}{Gorllewinol, Addysgedig, Diwydiannol, Cyfoethog a Democrataidd}
Mae cwestiwn ieithoedd isadnodd yn y DAmA yn gysylltiedig yn agos â'u sefyllfa gyffredinol ym maes prosesu iath naturiol (PIN). Disgrifir y sefyllfa yn orau yn \textcite{magueresse_low-resource_2020}, wrth i ddulliau ystadegol, cysylltiadol, ddod yn dominyddol mewn PIN, mae cwestiwn prinder data yn dod yn brif ffactor cyfyngol wrth gymhwyso atebion PIN modern i ieithoedd isadnodd. Mae'r broblem hon hefyd yn cael ei chymhlethu gan duedd WEIRD (am Wealthy, Educated, Industrialized and Democratic) cyffredinol mewn gwyddor wybyddol \parencite{henrich_most_2010}, lle mae ieithoedd o ddiwylliannau sy'n orllewinol, addysgedig, diwydiannol, cyfoethog a democrataidd yn tueddu i gael eu ffafrio ym mhob maes o'r gwyddor gwybyddol. Fodd bynnag, os mai'r ieithoedd isadnodd sy'n mabwysiadu'r technolegau hyn y lleiaf, yn eironig yr ieithoedd hyn y sydd â'r mwyaf i'w golli o beidio â'u mabwysiadu. Gall peidio â mabwysiadu'r technolegau hyn achosi colli gwelededd, parch a dymunoldeb, sy'n ei dro yn arwain at lai o fabwysiadu a defnydd, gan arwain at gylch dieflig lle mae llai o adnoddau hyfforddi ar gael i addasu'r technolegau hyn i ieithoedd isadnodd. Disgrifiwyd y ffenomen hon fel tranc digidol, neu farwolaeth, iaith, sef llofnod ar-lein ieithoedd sydd wedi darfod yn gymdeithasol \parencite{kornai_digital_2013}.

Ni ellir tanbrisio'r rôl y gallai technolegau addysgol ei chwarae wrth dorri'r cylch dieflig hwn, o leiaf ar gyfer rhai o'r ieithoedd dan sylw. Ar y naill law, gall helpu i addasu technolegau addysgol presennol i ieithoedd isadnodd helpu i gynnal eu perthnasedd fel cyfrwng dysgu i rieni sy'n dymuno'r safonau addysgol gorau i'w plant a chynnig dewis arall i bobl sy'n ceisio cyflawniad deallusol yn hytrach na gadael eu prifiaith yn syml i barhau i ddysgu pethau newydd. Deallir y gall technolegau PIN fel cyfieithiad awtomatig helpu i gario technolegau addysgol sefydledig i nifer fawr o gymunedau ieithyddol nad oes ganddynt yr adnoddau i ddatblygu eu hofferynau addysgol eu hunain fel arall \parencite{haddow_survey_2022}. Mae astudiaeth gan \textcite{horbach_crosslingual_2024} yn cefnogi'r syniad y gellir cyflawni cydraddoldeb addysgol trwy systemau sgorio traws-ieithog, yn y cyd-destun lle defnyddir cwestiynau agored i asesu sgiliau, a lle gall cefndiroedd ieithyddol gwahanol effeithio ar rugledd atebion myfyrwyr beth bynnag yw eu dealltwriaeth o'r cysyniad a asesir. Ar y llaw arall, o ran technolegau addysgol sy'n gyfeirio at iaith, mae'r maes bron yn gyfan gwbl dan dominyddiaeth ymchwil i ddysgu Saesneg, a hyd yn oed yn dod i gystadlu ag ieithoedd sydd eisoes mewn perygl. Mae papur gan \textcite{henkel_supporting_2025} yn symptomatig o'r periglau hynny. Yn yr astudiaeth hon, defnyddir technolegau adnabod lleferydd Saesneg mewn system DAmA i wella llythrennedd mewn ysgolion Ghana, gwlad sy'n gartref i fwy na 70 o ieithoedd brodorol \parencite{noauthor_ghana_nodate}.

Hyd y gwyddom, ymddengys mai ychydig o ymdrech a wnaed yn y llenyddiaeth academaidd i gefnogi datblygu technolegau addysgol wedi'u teilwra'n benodol ar gyfer anghenion ieithoedd isadnodd a'u cymunedau siaradwyr, er gwaethaf yr holl gynnydd a wnaed yn y blynyddoedd diwethaf i ddatblygu'r ieithoedd hyn mewn PIN\@. Gallai'r diffyg tystiolaeth hwn fod wedi'i achosi gan rwystr iaith, ond nid yw hyn ond yn atgyfnerthu'r syniad y dylid, os nad oes rhaid, gwneud mwy i gefnogi presenoldeb yr ieithoedd isadnodd yn y DAmA\@.

    \subsection{Deallusrwydd Artiffisial ac Addysg}
    \abbrv{DD}{Dysgu Dwfn}
    \abbrv{DA}{Deallusrwydd Artiffisial}
    \abbrv{GOFAI}{Good Old-Fashioned AI (Ddeallusrwydd Artiffisial Da Hen-Ffasiynol)}
Fel y dangoswyd gan \textcite{doroudi_intertwined_2023}, bu rhwng ddeallusrwydd artiffisial (DA) ac ymchwil mewn addysg ddeialetig 70 mlynedd o hyd a fu o fudd i'r ddau faes hyn. Os tynnodd gwaith cynnar ar DA o seicoleg ddatblygiadol yn wreiddiol a hyd yn oed ddatblygu offer addysgol fel rhan o'u hymdrech i efelychu deallusrwydd dynol gyda pheiriannau, maes addysg sydd bellach yn elwa o'r posibiliadau a ddatgloir gan dechnolegau DA modern.

Archwiliodd ymchwil gynnar mewn deallusrwydd artiffisial ddau ddull gwahanol i geisio efelychu prosesau gwybyddol. Gelwir y cyntaf yn gyffredin fel Good Old-Fashioned AI (GOFAI), roedd wedi'i ganoli o amgylch dull symbolig a ddeilliodd o waith semenaidd Allen Newell, Herbert A. Simon a Cliff Shaw ar y Logic Theorist \parencite{newell_logic_1956}. Ceisiodd y dull hwn ddeall sut mae arbenigwyr yn datrys problemau gan ddefnyddio systemau sy'n seiliedig ar reolau a haniaethu symbolig. Roedd yr ail ddull, cysylltiadol, wedi'i ganoli o amgylch rhwydweithiau niwral ac yn canolbwyntio ar y prosesau caffael sgiliau gwybyddol dros berfformiad priodol. Yn y byd Cymraeg, mae Cysill \parencite{hicks_welsh_2004} yn engraifft o system DA seiliedig ar reolau. Datblygwyd GOFAI gan bobl fel Marvin Minsky, Seymour Papert a llawer o rai eraill \parencite{doroudi_intertwined_2023}. Yn nodedig, daeth Seymour Papert i'r byd DA ar ôl astudio datblygiad gwybyddol plant yn labordy Jean Piaget yn Geneva. Daeth â dylanwad sylweddol o adeiladaeth (constructionism) Piaget i'r paradigm cysylltiadol mewn DA, sy'n ddamcaniaeth a esyd fod dysgwyr yn adeiladu eu sgiliau newydd a'u dealltwriaeth ar y wybodaeth a'r sgiliau a gafwyd eisoes.

Arweiniodd y ddau ddull at ymdrechion i greu systemau addysgol rhyngweithiol yn gynnar. Mae enghreifftiau o raglenni meddalwedd addysgol cynnar sy'n seiliedig ar GOFAI yn cynnwys system GUIDON, a oedd yn dibynnu ar beiriant Mycin, system ddiagnoseg haintiau, i ddysgu myfyrwyr i ddiagnosio patholegau \parencite{william_j_guidon_1983}. Ffafriodd y gangen gysylltiadol ddatblygu ``micro-bydoedd'' addysgol, megis ieithoedd rhaglennu addysgol, lle gallai plant ddysgu sgiliau datrys problemau anniffiniedig. Mae enghreifftiau o'r dull hwn yn cynnwys iaith raglennu Logo, a ddyluniwyd i ddysgu am leoli cymharol a geometreg trwy ddylunio rhaglenni i arwain crwbanod robot (lluniadu). Dilynodd llawer o systemau o'r fath, megis iaith raglennu Scratch a chitiau Lego Mindstorms. Ond arweiniodd yr arbenigedd angenrheidiol mewn DA at ymchwil ddiweddarach yn canolbwyntio'n llwyr ar berfformiad systemau cyfrifiadurol, yn enwedig wrth i ddyfodiad ôl-ledaeniad (back-propagation) roi bod i ddysgu dwfn (DD), gan lwyddo i sefydlu goruchafiaeth y paradigm cysylltiadol mewn DA\@.

Ar y pwynt hwn, symudodd y ffocws yn bendant o ddefnyddio seicoleg ddatblygiadol i gefnogi DA, i integreiddio atebion technegol DA mewn offer addysgol. Mae meta-ddadansoddiad gan \textcite{schmid_meta-analysis_2023} bellach yn cefnogi manteision dulliau addysgol adeiladol fel Dysgu Cyfun (Blended Learning) a'r Ystafell Ddosbarth Wedi'i Throi (Flipped Classroom), y sy'n rhoi mwy o rôl hyfforddi i'r athrawon, gyda chyfrifoldeb y cyfarwyddyd yn cael ei drosglwyddo i systemau rhyngweithiol ar-lein, a ddefnyddir allan o'r ddosbarth.

Yn yr adran hon, gwelwyd sut y llifodd syniadau adeiladol Piaget ar addysg yn y ddull gysylltiadol at y DA trwy waith Seymour Papert. Yna, pan enillodd y dydd y ddull gysylltiadol gyda dyfodiad y DD, daeth y DA yn ôl i'r addysg ar ffurf platfformau dysgu addasol i gefnogi datblygiad arferion adeiladol mewn ysgolion. Mae dysgu am yr hanes cyfun hwn yn mynd y tu hwnt i ymholiad am straeon hanesyddol yn unig, mae'n rhoi inni'r cwmpas a'r fframwaith epistemolegol i bennu nodau a dulliau technolegau addysgol, sy'n gam angenrheidiol i sicrhau y gallai sistemau addysgol newydd o'r fath gyflawni llwyddiant byd-go-iawn rywbryd. Hynny yw, nid fel system ynysig sy'n esblygu mewn gwactod, ond fel offer yng ngwasanaeth amgylchedd dysgu cyfannol.
    
    \subsection{Addasrwydd a Modelau Gwybodaeth}
        \subsubsection{Addewid Addasrwydd}
Y gwahaniaeth allweddol rhwng gwersllyfrau clasurol neu addysg sy'n seiliedig ar ddarlithiau a'r rhan fwyaf o'r technolegau dysgu diweddar yw addewid addasrwydd. Mae hyn yn golygu bod y system yn addasu ei hymddygiad yn seiliedig ar berfformiad y dysgwyr, yn ddelfrydol gyda'r nod o fwyhau eu derbyniad dysgu. Yn y rhan fwyaf o systemau modern, ond nid pob un, gwneir y mwyhaad hwn gan system argymell, y mae ei ffurfiau mwyaf soffistigedig yn datrys enghraifftiau o broblem y \href{https://en.wikipedia.org/wiki/Multi-armed_bandit}{bandit aml-fraich}. Problem y gellir ei datrys gan un o sawl algorithm gwahanol \parencite{chen_recommendation_2017}. Problem y bandit aml-fraich y sydd yn fformiwleiddiad mathemategol o sefyllfa lle cynigir gweithredoedd gwahanol, yn ein hachos ni, argymell deunyddiau dysgu gwahanol gyda gwerthoedd pedagogig ansicr, a rhaid i asiant benderfynu pa weithredoedd fydd yn mwyhau rhyw wobr benodol, neu fetrig, yma, twf y myfyrwyr mewn gwybodaeth. Rhaid i'r systemau hyn gyflafareddu rhwng ecsbloetio gweithredoedd gyda gwobrau hysbys, ond cyfyngedig ac archwilio gweithredoedd gyda gwobrau anhysbys.

Mae'r paradig hwn yn caniatáu i ddylunwyr systemau ryddhau eu hunain o'r cur pen a achosir gan orfod cyflafareddu'r cwestiwn sy'n ymwneud â dewis deunydd dysgu, fel eu hanhawster cymharol; un yn union ar bar â lefel y myfyriwr, neu un sy'n manteisio ar baragimau dysgu eraill fel \href{https://en.wikipedia.org/wiki/Desirable_difficulty}{anhawster dymunol}, neu ryw gyfuniad o'r ddau. Yn dibynnu ar yr algorithm a ddewiswyd, addewid dysgu addasol yw galluogi adeiladu proffil unigolyddol o sgiliau'r dysgwyr, o bosibl hefyd yn cynnwys disgrifiad o'u gallu neu rhythm dysgu, a chael y system i adeiladu cwricwlwm wedi'i optimeiddio i gyrraedd y nod pedagogig penodedig.

Rhaid nodi bod systemau sy'n seiliedig ar reolau dal yn cael eu peiriannu, ac yn cael eu gweithredu'n eang, lle mae'r cwricwlwm yn cael ei ddylunio ymlaen llaw yn seiliedig ar fodel pedagogig (yn aml, model pedagogig y system ysgolheigion), gan chwarae rôl y systemau argymell a gyflwynwyd uchod. Gall y systemau hynny fod yn berthnasol pan mai'r nod yw dysgu setiau penodol, diffiniedig o sgiliau, fel rhaglenni ysgolion cynradd ac uwchradd. Mae \textcite{pelanek_adaptive_2025} yn crybwyll y platfform \textit{Umíme} yng Ngweriniaeth Tsiec, sy'n ymddangos fel ei fod wedi'i fabwysiadu'n helaeth gan ysgolion yr wlad ac yn dibynnu ar bensaernïaeth o'r fath. Efallai bydd gan systemau addisgol eraill priodweddau rhyngweithiol yn unig, heb systemau argymell, fel yr ieithoedd rhaglennu addysgol a grybwyllwyd uchod, ond nid y rheiny yw ffocws y gwaith presennol.

        \subsubsection{Model Gwybodaeth a Nod Offerynnol}
Lle gall systemau argymell wneud yr addewid i optimeiddio unrhyw gyfarwyddyd penodedig, o amser gwylio fideo YouTube i weithgynhyrchu clipiau papur \parencite{bostrom_ethical_2003}, nid yw systemau DA yn dwyn y cyfrifoldeb i ddiffinio'r cyfarwyddiadau cyfryngol hyn, yr hyn a elwir yn nod offerynnol. Mae'r cwestiwn hwn wrth wraidd pob ystyriaeth aliniad, ac nid yw'r systemau addysgol yn ddieithr i'r broblem hon. Mewn systemau addysgol, mae'r procsi hwn yn seiliedig ar fodelau gwybodaeth, a elwir hefyd yn fodel myfyriwr, sy'n ddata seicometrig y gellir deillio ohono fodel dysgu (y datblygiad o'r wybodaeth honno ar y amser) ac a all yn ei dro gael ei ddefnyddio i ddiffinio gwerth pedagogig deunydd dysgu, y metrig hwn yw'r wobr y byddai algorithmau yn cael eu cyhuddo o'i optimeiddio. Mae diffinio'r model gwybodaeth hwn a natur y lluniad seicometrig y mae'n ei gasglu yn hanfodol i lwyddiant system dysgu addasol, ac mae'r diffiniad hwn yn gyfrifoldeb y maes y bwriedir i'r system ei ddysgu a modelau seicolegol, nid y DA yn uniongyrchol.

    \subsection{Casgliad}
Trwy'r adran hon, dadansoddwyd hanes technolegau addysgol ers y chwyldro gwybyddol yn y 1950au. Gwelwyd y potensial amhrisiadwy'r DAmA sy'n dal i ddatblygu, a'i addewid o addasrwydd, ynghyd â'r risgiau a'r cyfleoedd y mae'n eu dod i ieithoedd isadnodd. Nodwyd bwlch yn y llenyddiaeth academig ar ddysgu ieithoedd isadnodd yn y DAmA. Os gall cyfieithu systemau DAmA i ieithoedd isadnodd weithio cyn belled â bod y pwnc y bwriedir iddo ei ddysgu ddim yr iaith ei hun, o ran dysgu ieithoedd, mae'n ymddangos bod arloesedd yn y maes wedi'i ddominyddu gan Saesneg, iaith sy'n eithriad o ran argaeledd adnoddau o'i gymharu â mwyafrif y 7000 o ieithoedd eraill a siaredir ledled y byd. Yn y cyd-destun hwn, mae'n ymddangos yn angenrheidiol aillfeddwl sut y gellir cyflawni addasrwydd pan nad oes gan y rhan fwyaf o ieithoedd y byd hyd yn oed ramadeg ddisgrifiadol wyddonol, heb sôn am y dwsinau o oriau o recordiadau wedi'u hanodi sy'n angenrheidiol i hyfforddi systemau adnabod lleferydd.