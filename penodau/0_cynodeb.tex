\abstract{Mae'r traethawd hir hwn yn cyflwyno leksis, prawf geirfa tasg penderfyniad geirfaol. Mae'r mecanwaith sgorio yn deillio o system raddio Elo, a ddefnyddir mewn gwyddbwyll, sy'n gweithio ar yr un egwyddorion â model Rasch sy'n fwy cyffredin mewn seicometrig. Mae sawl fersiwn o'r profion wedi'u gweithredu hyd yn hyn, un Llydaweg yn gyntaf, yna eraill ar gyfer Cymraeg, Ffrangeg ac Wcreineg. Nod y traethawd hir yw mesur addasrwydd y profion. Cyfaddefir mai pwrpas y prawf yw dod â'r sawl sy'n cymryd y prawf i'r pwynt lle mae'r siawns o adnabod gair go iawn yn union ar 50\%, hynny yw, yn gwbl ansicr. Ar ôl ychydig mwy na dau fis, mae'r canlyniadau a gasglwyd ar gyfer y prawf Llydaweg yn ymddangos yn bendant dros sawl ystod lefel geirfa. Mae'r ystodau gyda'r mwyaf o ddata a gasglwyd yn ymddangos wedi'u calibro'n dda, sy'n ymddangos yn atgyfnerthu'r syniad nad oes angen llawer o ymdrech calibro i greu prawf dibynadwy. Yr arloesedd sy'n gwneud y gamp hon yn bosibl yw dull a alwyd gennym yn glystyru "ffa" neu "modulo". Mae'r dechneg gychwyn graddio eitemau hon yn effeithiol yn troi unrhyw brawf cwbl addasol sy'n seiliedig ar Elo yn system hybrid rhwng cyfran o system sy'n seiliedig ar atebion cywir a graddfa logistaidd y dylai fod yn y lle cyntaf, gan ddod o hyd i gyfaddawd rhwng calibradu eitemau newydd (graddadwyedd) ac ecsbloetio eitemau hysbys (manylder). Mae'r prawf hefyd yn cyflwyno'r defnydd o RNN (LSTM) wrth wneud geiriau ffug, a oedd yn ymddangos yn argyhoeddiadol, gyda phumed ran ohonynt yn cael eu hadnabod fel geiriau go iawn ar gyfartaledd gan y rhai a gymerodd y prawf.
Ar ddiwedd sesiwn, mae'r prawf yn creu awgrym dadansoddi personol y gall y rhai a gymerodd y prawf ei rannu gydag LLM i gynhyrchu gwersi iaith rhyngweithiol yn seiliedig ar y canlyniadau. Offeryn y mae ei werth addysgeg yn dal i gael ei asesu.}