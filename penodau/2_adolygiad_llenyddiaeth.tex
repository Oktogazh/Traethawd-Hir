\chapter{Adolygiad Llenyddiaeth}

Mae'r adolygiad llenyddiaeth hwn wedi'i rannu'n ddwy brif adran. Mae'r adran gyntaf yn canolbwyntion ar ddadansoddi'r lluniadau a archwiliwyd wrth asesu hyfedredd iaith, ac ymhlith y rheini, pa rai a allai weithio mewn system ddysgu addasol, tra bo'r ail adran yn ymdrîn â'r ffyrdd ystadegol i sgorio rhyw luniad penodol. Y brif-egwyddor trwy'r bennod hon fydd symlrwydd y datrysiadau a gynigir, gan mai bob amser yw'n haws trwsio diffygion systemau syml na rhai systemau cymhleth.

\section{Lluniadau'r Hyfedredd a Ble i'w Canfod}
\abbrv{CAI}{Caffael Ail Iaith}
\abbrv{GD}{Gwyddor Dysgu}

Dangosodd y rhagymadrodd fod diffiniad yr amcanion offerynnol y mae’n rhaid i system argymell (recommender system) eu hoptimeiddio yn perthyn i’r faes arbenigedd sy’n ymwneud â phrif nod y system, yn hytrach nag i’r dechnoleg ei hun. Mae profi iaith yn draddodiadol wedi bod yn fater ymchwil Caffael Ail Iaith (CAI), y gellir ei hystyried fel is-faes gwyddor dysgu (GD), ond mae'r maes hwn yn derbyn mewnbwn gan – ac yn perthyn yn agos i – seicoieithyddiaeth, ieithyddiaeth gymhwysol, ac fel y gwelwn, niwrowyddoniaeth. Dibynnir ar y rhain am ddealltwriaeth
gyffredinol o’r prosesau sy’n gysylltiedig â defnydd a chaffael iaith. Heb honni bod yn adolygiad cynhwysfawr, bydd yr adran hon yn ceisio darparu trosolwg o hyfedredd iaith a'r ffyrdd i'w fesur.

Sut mesuro hyfedredd iaith? Mae'r cwestiwn hwn wedi cael ei astudio'n eang o fewn fframweithiau damcaniaethol amrywiol ac ar gyfer sawl pwrpas ymarferol. Y fwyaf pwysig efallai, ydy bod myndeiad i bethau fel dinasyddiaeth, addysg neu swyddi newydd yn dibynnu ar feistrioliaeth iaith. Mynediadau sy'n gyfleoedd llunio bywyd, sydd wedi gwneud ei dilysiad yn fater symudedd cymdeithasol. Yn yr adran hon, cyflwynir y ffordd cyffredinolaf ac sydd yn cael ei defnyddio mwyaf i asesu sgiliau iaith, cyn symud tuag at atebion amgen a fyddai'n cyd-fynd ag anghenion system ddysgu addasol raddadwy. Yn olaf, asesir yr atebion amgen hyn yn feirniadol. Mae'r ail adran yn canolbwyntio ar ddod o hyd i ffyrdd i fynd i'r afael â diffygion yr atebion amgen hyn.

\subsection{Y Dullau Cyfannol o Brofi (CEFR)}
\abbrv{CEFR}{Fframwaith Cyfeirio Cyffredin Ewropeaidd ar gyfer ieithoedd (Common European Framework of Reference for languages)}

Gellir asesu nodweddion cudd cymhleth fel hyfedredd iaith gan ddau baradigm profi, y cyntaf yn cael ei ddisgrifio fel uchafsymiol, cynhwysfawr neu gyfannol, a'r ail fel lleiafsymiol, ar sail-procsi neu leihaol. Mae profion iaith masnachol a sefydliadol megis yr IELTS a Chymwysterau Saesneg Caergrawnt ar gyfer Saesneg neu'r DELF a DALF ar gyfer Ffrangeg, yn dilyn dull uchafsymiol a ddiffinnir gan y Fframwaith Cyfeirio Cyffredin Ewropeaidd ar gyfer ieithoedd (CEFR) \parencite{europe_common_2020}. Nid yn unig y mae'r fframwaith hwn yn diffinio'r chwech gradd alffarifol enwog bellach o feistrolaeth iaith, ond hefyd y pedwar cyd-destun defnydd y dylid eu mesur ynddynt. Y pedwar hyn sydd yn gyfuniadau o ddau ddull defnydd, llafar ac ysgrifenedig, ar gyfer dau fath o weithgareddau, derbyn a chynhyrchu. Mae'n mesur y wybodaeth ieithyddol (geirfa, gramadeg a'u cydrannau) yn syth trwy'r pedwar sgîl iaith y mae defnyddwyr iaith yn ymgysylltu â hwy yn ymarferol: gwrando, siarad, darllen ac ysgrifennu. Ystyrir y fframwaith hwn yn safonol y tu hwnt i ffiniau Ewrop, ond er gwaethaf ei gryfderau, efallai na fydd yn addas ar gyfer anghenion profi pob iaith.

Y brif feirniadaeth y gellid ei chodi yn erbyn y paradigm profi hwn yw'r ffaith mai dim ond deg iaith Ewropeaidd sy'n gallu ymfalchïo bod ganddynt brofion sy'n cydymffurfio â CEFR ac sy'n cynnwys y chwech lefel hyfredeedd y mae'n eu diffinio \parencite{noauthor_common_2025, noauthor_cadre_2025}. Ar ôl pum mlynedd ar hugain o fodolaeth, nid yw hyd yn oed ieithoedd cenedlaethol economïau arweiniol yr UE fel Iseldireg neu Tsieceg yn perthyn i'r rhestr hon. Mae hwn yn ddiffyg sylfaenol ar gyfer paradigm a ddyluniwyd yn benodol i beidio â ffafrio prif ieithoedd yr Undeb. Mae'r rhesymau am hyn yn amlwg, dim ond yr ieithoedd mwyaf ``marchnadadwy'' sy'n gallu datblygu ecosystem addysgol digon cryf i wneud y profion hyn yn economaidd hyfyw. Weithiau, fe gall ewyllys wleidyddol pontio'r bwlch fel ar gyfer ieithoedd rhanbarthol yn Sbaen (mae Galisieg a Chatalaneg ymysg y deg iaith a grybwyllwyd uchod, pan mai dim ond prawf ar gyfer lefelau A1–2 sy'n eisiau ar Fasgeg), ond adeiladir yr ewyllys hon ar sefydliadau ac arbenigedd cryf sydd gan ddim ond llond llaw o ieithoedd ar gael iddynt yn Ewrop, heb sôn am weddill y byd. Er gwaethaf ei sylfaen ddamcaniaethol, mae prinder yr adnoddau (amser, arian, arbenigedd a diddordeb gwleidyddol) yn gwneud y paradigm cynhwysfawr profi ieithoedd yn anymarferol ar gyfer y rhan fwyaf ohonynt, ac y rhain sydd unwaith eto’n cael eu gadael ar ôl. Unwaith eto, yr ieithoedd sydd â'r mwyaf i'w ennill o'r offer hyn, a'r mwyaf i'w golli trwy beidio â'u defnyddio, sy'n wynebu'r anawsterau mwyaf i gael mynediad atynt. Eto, yn achos system ddysgu addasol, sef y prif bwrpas ar gyfer y traethawd hwn, byddai prawf cynhwysfawr rhy hir ac yn diangen, gan y byddai'r profi yn cymryd gormod o amser o'r profiad dysgu, oni bai fod y profi yn rhan o'r addysgeg ei hun.

Rhaid inni edrych felly ar ffyrdd mwy effeithlon o fesur hyfedredd. Ond cyn hyn, mae angen inni ddatblygu dealltwriaeth ddyfnach o beth mae caffael iaith yn ei olygu. Sut y mae'r wybodaeth ddamcaniaethol haniaethol sy'n bresennol yn y geiriaduron a'r gramadegau nas darllenwyd gan neb yn cysylltu â'r dau neu bedwar sgîl ymarferol sy'n hysbys ledled y byd.  Beth yw cymhwysedd a pherfformiad pan sonnir am hyfedredd iaith?

\subsection{Natur gymhleth y syniad o Hyfedredd}
Mae'r rhan fwyaf o ddamcaniaethau mewn ieithyddiaeth, yn enwedig strwythuriaeth de Saussure a chynyrchioliaeth Chomsky, yn seiliedig ar ddull dadansoddol. Gan gymryd iaith ar wahân yn gyntaf o brosesau gwybyddol eraill, yna gwahanu ei chydrannau cysyniadol, geirfa oddi wrth ramadeg, cymhwysedd oddi wrth berfformiad \parencite{chomsky_aspects_1965} a chadw ailadrodd y broses gyda'u cydrannau ac is-gydrannau. Ac yna, astudio'r ffyrdd i'w cyfuno gyda'i gilydd. Mewn ffordd, mae paradigm CEFR yn dilyn yr un tuedd epidemiolegol ``dadansoddol'', drwy rannu sgiliau cynhyrchu a chanfod, yn ogystal a defnydd llafar ac ysgrifenedig. Mae prif fantais y dulliau dadansoddol hyn yn amlwg, trwy wahanu agweddau a chategorïau, gall rhywun gyflawni dealltwriaeth gynhwysfawr o gydrannau a rheolau systemau cymhleth megis ieithoedd. Ond er gwaethaf ei gryfder, mae'r dull dadansoddol hwn yn dod â golwg ragfarnllyd o beth yw iaith, gan ei fod yn dod â llun statig ac ynysig i'r systemau y mae'n eu hastudio. Fodd bynnag, nid yw ieithoedd, neu ran hynny gwybodaeth ieithol, byth yn strwythur gwbl statig nac yn olyniad o gyflyrau synchronig, oherwydd mae'r ieithoedd yn byw mewn cnawd dynol, mae rhaid iddynt gael eu caffael a'u hanghofio gan bob cenhedlaeth sy'n mynd heibio ac nid ydynt byth yn llonydd, nac yn gyfyngedig i'w strwythur mewnol. Dyma ble mae dulliau modern, fel swyddogaethiaeth neu ieithyddiaeth wybyddol \textcite{evans_cognitive_2009} yn dod i mewn i'r darlun, ynghyd â seicoieithyddiaeth ddatblygiadol, trwy ddod â'r ffocws i gaffaeliad a defnydd yr iaith a'i berthynas â'r corff, yn hytrach na'i strwythur. Dadleua \textcite{bybee_usage-based_1999} y gall ieithyddiaeth ``ddefnydd-seilig'' gynhyrchu modelau ffurfiol, ond gyda thro. Trwy ddatgan bod y gymhwysedd yn dod fel rhan o ddefnydd, bron fel priodoledd allddodol\footnote{Emerging property}, a'r defnydd hwn o'r iaith yn weithgaredd cymdeithasol, corfforol a gwybyddol yn bennaf, mae'r paradigm newydd hwn yn dod ag ystyriaethau newydd i'r golwg. Lle mae cynhyrchioliaeth yn gweld perfformiad fel gwireddu strwythurau cynhenid yr ymenydd gan roi blaenoriaeth i'r strwythur dros unrhyw beth ieithyddol, mae dulliau sy'n seiliedig ar ddefnydd yn ystyried strwythurau fel cyffredinoliadau a wneir gan yr ymennydd a ddysg iaith. Mae'r safbwynt hwn yn mynd y tu hwnt i wrthdroi blaenoriaeth. Trwy bwysleisio bod prosesau gwybyddol bob amser â rhyw radd o ddibyniaeth ar brosesau corffored, synhwyraidd-weithredol, mae'r safbwynt hwn hefyd yn torri deuoliaeth meddwl-corff Descartes \parencite{varela_embodied_1991} yn ogystal â deuoliaeth gymhwysedd-perfformiad Chomsky. Mewn geiriau syml, mae popeth yn yr ymennydd wedi'i gysylltu (neu'n dod i fod wedi'i gysylltu yn y pen draw) yn seiliedig ar ddefnydd, ac mae strwythurau bob amser yn dod a posteriori.

Mae'r datblygiadau hyn mewn ieithyddiaeth ynddi hefyd yn cael eu cefnogi gan ddatblygiad diweddar mewn niwroleg. Ers eu darganfyddiad gan Vermon Mountcastle yn y 1950au, bu dadlau ai yw'r colofnau cortigol sy'n strwythuro'r deunydd llwyd yn y neocortecs yn chwarae rôl fel uned fodiwlar cyfrifiant \parencite{horton_cortical_2005}. Rhagdybiaeth y mil ymennydd (Thousand Brains Hypothesis) \parencite{hawkins_theory_2017, hawkins_thousand_2021} yw'r iteriad diweddaraf o'r syniad hwn. Mae'n cynnig model ar sut y gall y pensaernïaeth unigryw hon, trwy fecanweithiau pleidleisio, fapio ysgogiadau synhwyraidd-weithredol yn raddol tuag at ac oddi wrth wahanol raddau o haniaethu. Ac ati i fireinio cynrychioliad unedig o'r byd, ac felly ymgysylltu'n well â'r byd mewn dolen adborth barhaus. Mae hyn yn cynhyrchu dadl gymhellol ar sut y gall meddwl haniaethol ac iaith ddod i'r amlwg yn raddol o ryngweithiadau synhwyraidd-weithredol \parencite{constantinescu_organizing_2016}, pan fo genynnau Gramadeg Cyffredinol Chomsky yn dal i aros i gael eu canfod yn unman.

\subsubsection{Goblygiadau ar gyfer Profi Iaith}
\abbrv{L1}{Iaith Gyntaf neu Brifiaith}
\abbrv{L2}{Ail Iaith (ieithoedd)}
Ar y pwynt hwn, rhaid egluro'r cyfatebiaeth rhwng paradigm profion y CEFR a ieithyddiaeth ffurfiol, oherwydd ym mharadigm y CEFR, mewn ffordd, rydym yn mesur perfformiad i ddidoli cymhwysedd, felly nid yw'r linc rhwng rheini yn cael ei gwadu. Ond mae'r feirniadaeth epistemolegol o'r ymchwil am gynhwysedd fel un sy'n tanseilio dealltwriaeth o ddynameg y broses caffael yn dal i sefyll. Os oes gennym ddiddordeb yn y broses caffael a'i ddynameg, mae cynrychioliad cyflawn, statig o'r sgiliau yn wrthgynhyrchiol. Ymhellach, os nad yw'r cymhwysedd yn bodoli'n annibynnol o'r perfformiad, a ellid didoli'r sgiliau o'r wybodaeth ei hun? Dyma y mae'n ymddangos bod ieithyddiaeth swyddogaethol yn dadlau drosto.

Os yw popeth yn gysylltiedig, os yw popeth yn un (er nad yw un yn bopeth), hynny yw, os yw mwy o ymarfer yn arwain at well sgiliau ymarferol, neu berfformiad, y sy'n arwain at well gwybodaeth ddamcaniaethol, neu gymhwysedd, yna, mewn theori, gallai perfformiad gael ei fesur trwy unrhyw luniad sy'n disgrifio cymhwysedd, megis gwybodaeth eirfa. Mae geirfa yn arbennig o ddiddorol gan fod ei chaffael yn broses ddisgret, ond sy byth yn gorffen yn ystod taith dysgu iaith. Cyhoeddodd \textcite{eun_hee_jeon_understanding_2022} gyfres o meta-ddadansoddiadau ar gydberthnasau'r gwahanol sgiliau ymarferol a ddiffinnir gan y CEFR, i gyd yn pwyntio tuag at y cyfeiriad hwn, gyda gwybodaeth eirfa yn cael ei dyfynnu fel cydberthynas gref ar gyfer rhuglder mewn gwrando \parencite{innami_meta-analysis_2022}, siarad \parencite{jeon_meta-analysis_2022}, darllen \parencite{jeon_updated_2022} ac ysgrifennu \parencite{kojima_meta-analysis_2022}. Sylwer fodd bynnag nad yw hyn yn golygu bod gwybodaeth eirfa yn achosi rhuglder, er ei bod yn cyfrannu ato i'r graddau na ddaw rhuglder heb lefel uwch o wybodaeth eirfa. Mae'r rhagdybiaeth sylfaenol hon yn agor y drws ar brofion cyflym â chynllun isel, cost isel, sy'n hygyrch i IAC ac a all fod yn fwy graddadwy a chymwys mewn llawer o feysydd, o hunanasesu, i ddatblygu systemau olrhain dysgu iaith awtomatig fel y grybwyllwyd yn y cyflwyniad. Yn nodedig, yng nghyd-destun IAC, y gall rhai eu galw'n ``ieithoedd llafar'', mae'r syniad bod lefel eirfa uwch yn gysylltiedig â sgil ymarferol yn dod yn fwy tebygol fyth, oherwydd mai'r ffordd bennaf o gael mynediad at wybodaeth yw ``defnydd mwy integredig'' (nid yw rhywun yn dysgu Rapa Nui yn y llyfrau). Yn y ffordd hon, gall rhywun hyd yn oed osod y gall profi geirfa ddod yn fwyfwy perthnasol wrth i lai o adnoddau ysgrifenedig a digidol fod ar gael i iaith benodol.

Goblygiad olaf y golwg prif-egwyddor a chysylltiadol hwn o gaffael iaith yw absenoldeb gwahaniaeth ymarferol rhwng y ffordd y ceir cymhwysedd mewn iaith gyntaf (L1) neu ail iaith (L2), hynny yw, trwy ddefnydd. Unwaith yr adeiladir y cylchedwaith sy'n gyfrifol am ddefnydd unrhyw iaith rhwng oedran 1 a 6, naill ai trwy addysg uniaith (gan gynnwys iaith arwyddion) neu amlieithog, mae'r ffordd y ceir geiriau newydd yn gyson ar draws yr ieithoedd a ddysgir yn ddiweddarach yn y fywyd. Os darganfyddir gair neu nodwedd trwy ddefnydd integredig ac mae'r darn gwybodaeth yn yr ymennydd yn tarddu o brofiad synhwyraidd sy'n bresennol yn ystod caffael y term, ac os dysgir gair yn L2 fel cyfieithiad gair yn L1, bydd ei gynrychioliad yn yr ymennydd yn tarddu o'r gair L1 fel ei gyfystyr o fewn ``cofrestr'' arall sef rhwydwaith yr L2. Mae'r ddau scenario yn awgrymu ffurfiant gwybodaeth o'r cyd-destun defnydd ond heb wahaniaeth mewn statws rhwng rhwydweithiau L1 ac L2. Gellir dysgu gair yn L2 fel cynnyrch profiad integredig, a gellir dysgu ei gyfwerth L1 yn ddiweddarach fel ``cyfystyr mewn gofod arall''. Fel rhywun a ddysgodd am ôl-ledaeniad, \textit{backpropagation}, yn Saesneg yn gyntaf, fy nhrydedd iaith, gallaf sicrhau'r darllenydd fy mod yn dal angen meddwl am y gair Saesneg cyn dod o hyd i'w gyfieithiadau yn ystod sgwrs yn Ffrangeg neu Lydaweg, a heb syniad am sut i siarad am hynna yn Gymraeg. Unwaith eto, mae'r gyfatebiaeth hon rhwng L1 ac L2 yn gyfleus yng nghyd-destun IAC, oherwydd bod yr ieithoedd hyn yn aml yn yr amrywiaeth isel mewn rhanbarthau deuglosig, lle mae'r lluniad o siaradwr brodorol a'r llinell rhwng L1 ac L2 yn aml yn aneglur.

\subsection{Topograffi y Profion Geirfa}
\abbrv{CDG}{Cynnyrch Domestig Gros}
Dangoswyd yn aml bod dewis procsiau yn ofalyn yn gallu rhoi darlun dibynadwy o brosesau cymhleth sy'n cael eu mesur. Mae economegwyr wedi dangos er enghraifft sut y gall mesur golau nos o'r gofod wasanaethu fel dangosydd twf CDG\footnote{Cynnyrch Domestig Gros, sef GDP} dibynadwy mewn gwledydd lle gall ystadegau swyddogol fod yn brin o ran ansawdd neu onestrwydd \parencite{henderson_measuring_2009}, hyd yn oed heb ddarparu rheswm pam y gallai hyn weithio. Dychmygodd ieithyddion lawer o ffyrdd i ddiffinio a mesur gwybodaeth eirfa, wrth iddynt ddeall a dangos y gydberthynas gref yr oedd ganddi â rhannau eraill hyfedredd iaith. Bydd y rhan olaf hon o hanner cyntaf yr adolygiad llenyddiaeth yn rhoi trosolwg o'r gwahanol ffyrdd y ceisiodd ieithyddion fesur geirfa hyd yn hyn.

\subsubsection{Profion Geirfa Gynhyrchiol}
Y ffyrdd mwyaf integredig o brofi geirfa yw gofyn i'r rheini sy'n sefyll y prawf roi cyfystyr rhyw eiriau, gan asesu felly sgiliau geirfa gynhyrchiol, nid yn unig o'r geiriau a ellir eu hadnabod a'u deall, ond hefyd eu hadfer o'u hystyr yn unig. Mae'n un o'r strategaethau a ddefnyddir i fesur y ``mynegai geirfa'' (vocabulary index), sy'n cael ei gyfuno â thri mynegai arall i gyfrifo IQ y person sy'n sefyll y profion yn sgraddfeydd deallusrwydd oedolion a phlant Wechsler (WISC neu WISA) \parencite{wechsler_wechsler_nodate}.

\subsubsection{Profion Geirfa Dderbyniol}
\abbrv{VLT}{Vocabulary Level Test – Prawf Lefelau Geirfa}
Yn ail y ceir cyfres o brofion sy'n anelu at fesur sgiliau geirfa dderbyniol, y geiriau y gellir eu cysylltu â'u hystyr gan y rheini sy'n sefyll y prawf. Y mwyaf defnyddiol o'r rheini yw'r Prawf Lefelau Geirfa (VLT, am \textit{Vocabulary Level Test}), a ddatblygwyd yn y 1980au gan \textcite{nation_teaching_1990} (gweler~\cite{kremmel_vocabulary_2017} am fwy o fanylion am ei weithrediad, datblygiad hannesyddol a chymhwysiadau). Dyluniwyd y prawf hwn ar gyfer defnydd eang mewn ysgolion fel prawf lleoli  myfyrwyr. Mae VLT yn addasol yn rannol hefyd, gan ei fod yn profi'r sgiliau i gysylltu termau sy'n gysylltiedig o ran ystyr o wahanol ystodau amlder. Dyluniad prawf geirfa dderbyniol diddorol arall yw'r brawf geirfa darluniol Peabody \parencite{dunn_ppvt-4_nodate}. Gan ei fod yn seiliedig ar luniau yn hytrach na geiriau ysgrifenedig, mae'n caniatáu profi plant na allent fel arall ddarllen y geiriau sy'n cael eu hasesu. Gallai'r dull sy'n seiliedig ar luniau hwn ymddangos fel pe bai'n gwneud y dyluniad profi hwn yn addas delfrydol i'w raddio yn blaenorol, ac felly'n addas ar gyfer safon gyffredinol y gellid ei chymhwyso hyd yn oed mewn amgylcheddau lle nad yw llythrennedd yn eang. Fodd bynnag, efallai bod y syniad hwn yn dda ar yr wyneb yn unig, gan fod y graddnodi ar gyfer mapio lluniau-geiriau wedi digwydd mewn gwlad sy'n siarad Saesneg, a gall y geiriau a ddefnyddir i ddisgrifio sefyllfaoedd tebyg amrywio'n fawr rhwng gwahanol ofodau ieithyddol. Dyma ddysgodd \textcite{kartushina_use_2022} y ffordd anodd wrth iddynt geisio cyfieithu'r prawf yn Rwseg ar gyfer plant cyn-ysgol, gan ddangos braidd yn ddamweiniol efallai mai prawf Peabody yw un o'r profion geirfa anoddaf i'w addasu i ieithoedd eraill, hyd yn oed rhai a siaredir mewn cymdeithasau sy'n braidd yn WEIRD fel Rwsia.

\subsubsection{Profion Geirfa Adnabod}
\abbrv{LDT}{Lexical Decision Task – Tasg Penderfynu Geirfaol}
\abbrv{SDT}{Theori Canfod Signalau}
Yn olaf, y teulu symlaf o brofion geirfa yw'r profion geirfa adnabod, a elwir weithiau'n ``brofion geirfa syml''. Maent yn mesur y gallu i adnabod presenoldeb gair cywir yn unig, heb fynnu cyfiawnhad o ddealltwriaeth bellach o ystyr y gair. Am drosolwg ac asesiad o wahanol ddyluniadau, gweler~\cite{meara_complexities_1994}. Y dyluniad mwyaf llwyddiannus o'r teulu profi geirfa syml hwn yw'r prawf geirfa seiliedig ar dasg penderfynu geirfaol (LDT am \textit{Lexical Decision Task}), rhoddwyd llawer o enwau eraill iddynt megis profion geirfa ``Ie/Na'' neu ``ddeuaidd'', ond mae'r cyfan yn dilyn yr un egwyddor; cyflwynir dilyniant o eitemau i'w profi, naill ai geiriau go iawn neu ffug-eiriau \parencite{meara_imaginary_2012} i'r rheini sy'n sefyll y prawf. Yna, gofynnir iddynt yn systematig a ydynt yn credu bod yr eitem dan sylw yn perthyn i eirfa'r iaith. Daw'r canlyniadau ar ffurf cyfuniad o'r pedwar allbwn a ddiffinnir gan fatrics dryswch\footnote{Gweld Confusion Matrix} (cyfuniad o atebion Cywir neu Anghywir a Phositif neu Negatif), sef trawiadau (PC), methu (NA), larwm ffug (PA) a gwrthod cywir (NA) ac mae gwahanol fethodolegau wedi cael eu cynnig i drin y canlyniadau, o dynnu canran yr atebion anghywir o ganran yr atebion cywir, hyd at gymhwyso systemau mwy cymhleth o Theori Canfod Signalau (TCS) \parencite{huibregtse_scores_2002}.

Mae llawer o brofion o'r fath wedi cael eu hadeiladu hyd yn hyn yn cynnwys o leiaf un fersiwn ar-lein, ar gael mewn sawl iaith, sef Saesneg, Iseldireg ac Almaeneg \parencite{lemhofer_introducing_2012}. Dangosodd y papur hwn ganlyniadau calonogol, gan ddangos gydberthynas gref o'r canlyniad geirfa gyda phrofion traddodiadol eraill, gan gefnogi felly'r syniad y gellir mesur rhuglder yn effeithiol trwy brofi geirfa. Mae prawf arall sydd, ar ei semlant, wedi cael ei greu ar gyfer Croateg \parencite{srce_how_2025}, er nad yw mwy o wybodaeth ar gael eto. Ac mae hyn ochr yn ochr â'r systemau niferus a ddatblygwyd gan Meara dros y blynyddoedd\cite{meara_complexities_1994}. Prif gyfyngiad y systemau hyn yw'r ffaith bod eu heitemau'n gyfyngedig a statig, felly ni chânt byth eu dylunio ar gyfer defnydd ailadroddus, a fyddai'n helpu mesur dynameg caffael geirfa. Mae hon yn broblem i'w datrys, oherwydd prif ddiddordeb prawf minimaliaidd yw caniatáu profion cylchol. Ar y llaw arall, ymddengys mai hwy yw'r profion byrraf i sefyll, yr hyn sy'n fantais pwysig.

\subsection{Perthnasedd a Chyfyngiadau Profion Geirfa}
Mae’r holl brofion geirfa a gyflwynwyd uchod wedi mwynhau rhyw raddau o lwyddiant masnachol neu academaidd oherwydd eu dibynadwyedd wrth ddal agweddau gwahanol ar gaffael geirfa. Rydym eisoes wedi esbonio'r rhesymau pam y dylai hyn fod felly yn adran 2.1.2. Os derbynnir bod unrhyw is-adeiledd o hyfedredd yn gysylltiedig yn yr ymennydd mewn ffordd a ddiffinnir gan ddefnydd, bod ``popeth yn un'', yna mae'r un rhesymeg yn berthnasol i eirfa. Mae adnabod yn dod fel y cam cyntaf yn ystod caffael geirfa. Heb adnabod, mae datblygiad ymhellach tuag at ddefnydd mwy itegredig yn amhosib. Mae'r holl deuluoedd profi hyn yn mesur gwahanol gamau yr un broses o gaffael geirfa. Nid yw mesur golau nos yn mesur y lluniad ``defnydd trydan nos-amser wedi'i neilltuo ar gyfer goleuadau stryd tiriogaeth'' yn unig, ond, fel y dangosodd yr ystadegau, gellir ei ddefnyddio fel dangosydd CMG, y sydd ei hun yn ddangosydd iechyd economaidd gwledydd. Yr un peth sy'n wir am y profion geirfa hyn, mae'r cyfan yn lluniadau sy'n mesur camau gwahanol yr un ffenomen caffael geirfa, sy'n rhan annatod o gaffael iaith.

Y brif wahaniaethau rhwng y profion hyn yw pa mor adnodd-ddwys ydynt a pha mor integredig yw'r lluniadau y maent yn eu mesur. Mae dangosyddion syml fel adnabod geirfa yn unig â gwendidau a gall fod yn destun twyllo neu lawdriniad. Honnwyd bod y mynegai enwog ``Big Mac'' yr Economist ar gyfer chwyddiant yn darged ymgaisiau llawdriniadau gan lywodraeth yr Ariannin yn 2011 \parencite{politi_argentinas_2011}, am yr union reswm hwn. Yn yr un modd, yr hawsaf yw caffael y sgil y tu ôl y lluniad a ddefnyddir fel dangosydd, y mwyaf tebygol y bydd yn dod yn destun ymdrechion i'w dylanwadu. Ond nid yw hyn yn golygu nad oes gan y lluniad unrhyw werth, yn wir, mae lefelau golau nos a phrisiau Big Mac yn dal i gael eu defnyddio heddiw, ond mewn meysydd a mewn cyd-destyn sy'n berthnasol i'w cymhlethdod. Yr un peth sy'n wir am seicometreg. Mewn cyd-destun profion geirfa, mae’r gofyniad dylunio cymhleth sy’n gorfod defnyddio llawer o adnoddau yn y Peabody Picture Vocabulary Test yn ei gwneud yn angenrheidiol cael defnydd masnachol i gefnogi ei ddatblygiad cymhleth. Mae'r profion eraill, symlach yn cyflawni dim ond llwyddiant academaidd oherwydd eu bod mor syml i'w rhoi ar waith fel na fyddant byth angen fasnacheiddio, sy'n cyfyngu ar eu potensial graddio ac yn ei dro eu datblygiad. Serch hynny, maent i gyd yr un mor ddefnyddiol wrth fesur eu camau priodol o gaffael geirfa, ac ar gyfer amcan y gwaith presennol, wrth fesur dynameg y proses caffael iaith.

\subsection{Casgliad}
Yng nghyd-destun profi awtomatig ac addasol gyda'r nod o olrhain caffael sgiliau iaith, mae manteision profion geirfa yn amlwg yn uwch na dulliau eraill. Yn eu plith, mae dyluniadau profion adnabod geirfa symlach yn wirioneddol ddisgleirio, yn enwedig wrth ystyried y broblem a osodir gan IAC\@. Mae profion geirfa LDT yn symlach i'w gweinyddu mewn ffordd gwbl awtomatig, ac maent yn haws eu trosglwyddo i IAC oherwydd gellir eu deillio o restr syml o gofnodion geiriadur. Eto, erys heriau sylweddol yn parhau cyn galluogi gweithrediad eang o brawf geirfa LDT\@. Y ffactor cyfyngu pennaf yw nifer yr eitemau a gynigir yn y profion fel LexTALE, pan oedd yn rhaid dewis geiriau go iawn a ffug-eiriau o set fwy mewn astudiaeth ragbaratol \parencite{lemhofer_introducing_2012}. Os yw prawf geirfa LDT i gael ei ddefnyddio mewn ffordd ddychweliadol, i olrhain cynnydd geirfa trwy amser, rhaid bod digon o eitemau ar gael, efallai'n cynnwys holl eirfa iaith neu o leiaf cyfran sylweddol ohoni. Ond felly, mae'r cwestiwn o raddnodi'r eitemau'n yn dod yn bwysig. Ni ellir ystyried defnyddio’r astudiaeth ragbaratol a wnaed ar gyfer dewis eitemau yn LexTALE i greu digon o eitemau i ganiatáu profi dychweliadol dibynadwy, hyd yn oed ar gyfer iaith â chyfoeth anhygoel o adnoddau fel Saesneg, heb sôn am ieithoedd adnoddau cyfyngedig. Byddai datrys y broblem hon o raddnodi'r eitemau yn agor y drws i raddio y math o bofion geirfa'n fertigol (caniatáu profi yn ddychweliadol yr un bobl) ac yn llorweddol (caniatáu cludo'r prawf i lawer o wahanol ieithoedd). Bydd yr adran nesaf yn cael ei neilltuo i ddod o hyd i ateb o'r fath.

\section{Olrhain Gwybodaeth}
\abbrv{OG}{Olrhain Gwybodaeth}
\abbrv{PAC}{Profi Addasol Cyfrifiadurol (neu CAT am Computerized Adaptive Test)}
I baraffrasu \textcite{meara_complexities_1994}, gall llawer o dasgau asesu fod yn ffyrdd dilys o asesu sgiliau adnabod geirfa, cyn belled â bod y dull priodol o ddadansoddi'n cael ei ddarparu. Mae'r adran hon yn neilltuo i'r broblem darparu hon. Tasg gymhleth yw mesur nodweddion cudd o ymatebion eitemau, pa un a adwaenir fel Olrhain Gwybodaeth (OG) \parencite{shen_survey_2024}. Cysyniad sylfaenol mewn Profi Addasol Cyfrifiadurol (PAC) yw cwestion dewis model OG dilys. Mae rhan o'r cymhlethdod hwn yn dibynnu ar y rhagdybiaethau y mae rhywun yn eu gwneud ar y nodweddion cudd, ai ydynt yn lluniad parhaus neu'n set o sgiliau arwahanol, sy'n cyfuno gyda'i gilydd mewn gofod gwybodaeth amldimensiwn, ac os felly, pa sgiliau sy'n dibynnu ar ba rai eraill? Gellir deffinio'r dimensiynau hyn a'r perthnasoedd rhyngddynt yn llawol neu yn seiliedig ar ddata, gan ddefnyddio technegau Bayesaidd neu rhywdweithiau niwral a DD\@. Gall rhagdybiaethau eraill gynnwys dylanwadau y broses brofi ar y broses ddysgu, ac yn yr achos hwnnw gall rhywun ystyried hanner-oes atgofion newydd a ffurfiwyd yn ystod rowndiau asesu blaenorol fel gan systemau ``flash-cards''. Yn sylfaenol, mae'r dewis cymhleth hwn o'r model yn arbitriad rhwng cywirdeb a dealladwyedd \parencite{pelanek_adaptive_2025}. Efallai fod modelau mwy ansoddol yn briodol i hysbysu argymhellion o adnoddau dysgu, ond cyflwyno canlyniadau prawf annibynnol fel fector rhuglder a fyddai yn galetach i'w ddehongli nâ sgôr glasurol. Problem arall yw, in fine, bod angen hyfforddiant aruthrol ar fodelau ansoddol i weithio, maent hywthau yn fodelau meintiol mewn fordd.

Gan fod y traethawd hwn yn canolbwyntio'n bennaf ar brofi twf geirfaol, mae mynegai unddimensiwn yn ymddangos yn briodolach. Ymhellach, byddai graddnodi paradigm amlddimensiwn yn gofyn am swm mawr o ddata neu adnoddau fel amser ac arbenigedd, nad ydynt ar gael ar gyfer IAC\@. Bydd diwedd y bennod hon yn gosod y sylfaen ddamcaniaethol ar gyfer y dehongliad meintiol hwn o ganlyniadau prawf geirfa LDT\@.

\subsection{Gallu Damcaniaethol Prawf Unddimensiwn Diswn}
Nod model olrhain gwybodaeth mewn PAC yw rhagfynegi peth byddai canlyniad profiad eitemau'r prawf er mwyn dewis yr eitemau y mae eu hatebion yn fwyaf ansicr, gan bob rhagfynegiad yn seiliedig ar ganlyniadau blaenorol. Yn jargon theori gwybodaeth, gelwir hyn mwyafu'r entropi, sy'n mwyafu'r elw gwybodaeth trwy leihau ansicrwydd y model. Gan dynnu o \textcite{shannon_mathematical_1948}, gall rhywun ddeffinio'n ddamcaniaethol y gallu diamod mewn prawf deuaidd diswn, cyn ei addasu i amgylchedd swnllyd. Mewn graddfa syml, unddimensiwn, gellir cyflawni dod o hyd i'r man ansicrwydd uchaf hwn gyda'r algorithm chwilio deuaidd. Mewn restr o eitemau wedi'u trefnu yn ôl anhawster, cymryd eitem yn y canol, ailadroddwch y broses gydag ail hanner y rhestr wreiddiol os yw'r ateb yn gywir (sef, bu'r eitem yn rhy awdd), fel arall, gyda'r hanner cyntaf. Ailadroddwch y broses nes bod y rhestr yn un eitem o hyd. Mae gan yr algorithm hwn gymhlethdod amser o $\theta(\log{n})$, sy'n golygu bod angen $\log_2(n)$ cam ar gyfer $n$ nifer o eitemau i gyrraedd yr eitem olaf. Dyma 10 eitem sydd angen eu profi ar gyfer graddfa sy'n cynnwys 1 024 o eitemau, 11 ar gyfer 2 048 eitem, 12 ar gyfer 4 096 ac yn y blaen\ldots

Gan dybio y gellid trefnu'r holl eiriau mewn geiriadur (bach) o 30 000 o eiriau yn ôl eu ``hanhawster'', a bod rhaid i hanner eitemau prawf fod yn ffug-eiriau i atal twyllo, byddai prawf seiledig ar yr algorithm hwn yn dod o hyd i lefel gyfredol y person sy'n sefyll y prawf mewn dim ond 30 rownd o brofi, i'w gymharu â'r 60 eitem a ddefnyddir gan brawf fel LexTALE \parencite{lemhofer_introducing_2012}. Hyd yn oed os ydym yn ystyried yr angen ar gyfer cywiro gwallau, bydd cyfanswm nifer y camau sydd eu hangen yn parhau i fod yn gyfrannol i'r dilyniant logarithmig hwn. Mae gan y gosodiad hwn gyfyngiadau amlwg ac elir i'r afael ganddynt yn yr isadran ganlynol, ond mae'n dod â dirnadiaeth diddorol ynghylch y broblem gwneud profion graddadwy. Yn bennaf, mae'n bosibl profi nifer mawr iawn o eitemau mewn ffordd effeithlon o ran amser, y peth sy'n agor y drws i ddefnyddio holl eirfa iaith, neu o leuaf rhan fawr ohoni, fel eitemau profi, yn hytrach na rhestr fer o eiriau detholedig. Mae'r posibilrwydd hwn yn ei dro yn agor y drws i profion unigryw, lle mae'r siawns o fynd ddwywaith trwy'r un geiriau mewn dwy brawf bron yn amhosibl. Mae hyn yn datgloi problem raddio fertigol a amlygwyd yn gynharach yn y bennod hon.

\subsection{Y System Graddio Elo}
\subsubsection{Graddio Elo a'r Model Rasch}
\abbrv{TYE}{Theori Ymateb Eitemau (neu IRT am Item Response Theory)}
\abbrv{IUa}{Iaith Uwchadnodd}
Cyfyngiad amlwg cyntaf y model a gynigiwyd yn flaenorol yw calibriad yr eitemau. Ni all rhywun gael lefel anhawster rhyw air yn uniongyrchol o eiriaduron, a gall y drefn y mae dysgwyr yn caffael geiriau amrywio'n fawr yn dibynnu ar wahanol ffactorau. Mae'r rhan fwyaf o brofion geirfa yn mynd o gwmpas y broblem hon drwy grwpio'r eitemau yn ôl ystodau amlder \parencite{nation_teaching_1990, meara_complexities_1994, dudley_context-aligned_2024}. Fodd bynnag, mae meddu ar restrau amlder yn aml yn fraint iaith uwchadnodd, ac nid oes gan y rhan fwyaf o IRA adnoddau o'r fath ar gael iddynt, neu os oes, ni fyddant mor gyflawned â'r rhain. Am y rheswm hwn, rydym yn cynnig y dylai graddio anhawster yr eitemau geiriau gael ei ddiweddaru'n uniongyrchol yn seiliedig ar ganlyniadau cynt y prawf.

Mewn profion safonol, cyflawnir y calibriad hwn o anhawster yr eitemau gan Theori Ymateb Eitemau (TYE), sy'n set o fodelau a ddeilliwyd o'r model Rasch \parencite{rasch_probabilistic_1980}. Ailddargaufuwyd y mathemateg y tu ôl i'r model Rasch sawl gwaith, gan gynnwys y tu allan i'r byd seicometrig, fel mewn gwyddbwyll gyda'r system graddio Elo \parencite{elo_uscf_1961, elo_rating_1986}. Cyflwynir yr hafaliadau allweddol ar gyfer y modelau hyn isod.

\begin{figure}[h]
    \centering
    \begin{minipage}{0.45\textwidth}
        \centering
            \[P(X_{AB} = 1)=\frac{1}{1+e^{R_B-R_A}}\]
        \captionof{figure}{Fformiwla Rasch}
    \end{minipage}
    \hfill
    \begin{minipage}{0.45\textwidth}
        \centering
            \[P(X_{AB} = 1)=\frac{1}{1+10^{\frac{R_B-R_A}{400}}}\]
        \captionof{figure}{System graddio Elo}\label{fig:Elo}
    \end{minipage}
\end{figure}

Yn y system graddio Elo, $P(X_{AB} = 1)$ sydd y tebygolrwydd y bydd chwaraewr gwyddbwyll $A$ o radd $R_A$ yn ennill trwy siachmat yn erbyn chwaraewr $B$ o raddio $R_B$. Yn y model Rasch, $P(X_{AB} = 1)$ sydd y tebygolrwydd y bydd rhyw berson $A$ gan radd $R_A$ wrth sefyll prawf yn ateb yn gywir eitem o radd anhawster $R_B$. Gan fod y ddau yn dilyn dilyniant logarithmig, gellir trosi unrhyw ``radd Rasch'' i radd Elo drwy ei luosi â $400/\ln(10)$ (neu $400\times\log_{10}(e)$) a gwrthdroi'r enwebwr a'r enwadur i fynd o Elo i Rasch. Roedd y gwahaniaeth yn sylfaen y logarithm a'r ychwanegiad o ``ffactor taeniad'' o 400 mewn gwyddbwyll yn golygu cynyddu darllenadwyedd a dehongliadwyedd, wrth gyfateb i systemau graddio a ddefnyddiwyd yn flaenorol yn y byd gwyddbwyll. Mae gwahaniaeth o 400 mewn graddio Elo yn golygu siawns buddugoliaeth o $1:11$ a $10:11$, sy'n fwy dehongladwy na gwahaniaeth o 1 pwynt yn golygu dosbarthiad siansiau o $0.2689:0.7311$ ac $0.7311:0.2689$.

Yn ymarferol, mae'r prif wahaniaeth rhwng y ddwy system yn gorwedd yn fwy yn y mecanweithiau diweddaru. Gan fod TYE wedi'i ddatblygu ar gyfer profion statig (heb nodweddion ymaddasu amser real), mae'n dibynnu ar dechnegau sy'n ddwys yn gyfrifiadol, nad ydynt yn addas iawn at ddiben PAC\@. Ei system ddiweddaru syml yw'r rheswm pam mae'r system graddio Elo wedi bod yn denu mwy o sylw yn y gymuned AIED dros y blynyddoedd. Mae \cite{pelanek_applications_2016} yn crybwyll nifer o integreiddiadau llwyddiannus o'r model hwn mewn gosodiadau addysgol ymaddasol, er byth ar gyfer profion annibynnol. Mae'r un erthygl hefyd yn cyflwyno mecanweithiau diweddaru amrywiol sy'n ystyried gwahanol tybiaethau, megis cywiriad wrth strategaethau twyllo neu hanner oes cof tymor byr a chanolig. Rhoddir diweddariad graddio Elo gan y fformiwla ganlynol.
\begin{equation}
    R_{A}^{\prime}= R_A+K \times{(S-P(S))}
\end{equation}\label{eq:Update Elo}
Mae'r sgôr gwirioneddol (1 neu 0) $S$ yn cael ei dynnu yr rhagfynegiad $P(S)$ yn seiliedig ar y gwahaniaeth sgôr a roddir yn y ffig.~\ref{fig:Elo}. Os yw canlyniad yn sicr (sef tua siansiau  1:0 gan mwy na gwahaniaeth graddio o 800) ac mae'r canlyniad yn dilyn y rhagfynegiad, bydd y gwerth hwn yn agos at sero a bydd y newid mewn graddio yn agos at 0. Os yw'r gwrthwyneb yn digwydd, mae'r sgôr yn cynyddu gan werth sy'n agos at $K$, a enwir yn ``ffactor K'', gwerth sy'n debyg i'r gyfradd ddysgu yn y byd DL. Gall y gwerth hwn amrywio yn dibynnu ar weithrediadau'r system graddio, ond yn aml mae oddeutu 20 yn y byd gwyddbwyll. Weithiau, defnyddir ``ffwythiant ansicrwydd'' i newid y gyfradd ddiweddaru yn raddol yn seiliedig ar nifer y diweddariadau cynt (cf.\ hafaliad~\ref{eq:uncertainty-function}).

\subsubsection{Cywiro Gwallau a Dirywiad}
\abbrv{HAdd}{Holiad Amlddewis}
Mae'r Holiadau Amlddewis (HAdd) yn defnyddio tri chategori o gydran, cwestionau neu \textit{ymholiadau}, am \textit{query} yn Saeseg, allweddi (yr atebion cywir) a llithiau, \textit{distractors} yn Saesneg, sef yr atebion anghywir. Yn sylfaenol, mae profion geirfa adnabod yn ffurf o HAdd, gydag ymholiad unigryw ar gyfer y prawf cyfan, a'r geiriau go iawn fel allweddi a'r ffug-eiriau fel llythiau. Cydnabyddir y gall fod rhesymau gwahanol pam y gall sefyllwyr prawf ddewis atebion cywir neu anghywir. Yr un mwyaf amlwg yw bod sefyllwyr prawf yn adnabod yr allweddi ac yn anwybyddu'r llythiau. Ond rhaid ystyried tri cwrs gweithredu arall.
\begin{enumerate}
    \item Mae'r sefyllwyr prawf yn gwybod yr ateb ond yn dewis ateb anghywir ar gam (e.e.\ drwy ateb yn rhy gyflym a sylwi ar y camgymeriad yn rhy hwyr).
    \item Nid yw'r sefyllwyr prawf yn gwybod yr ateb cywir, ac mae'n ateb yn briodol drwy siawns bur.
    \item Nid yw graddio'r eitem yn cyfateb i'w lefel anhawster gwirioneddol oherwydd nad yw'r calibriad ar ben (ac yn wir, ni all fod).
\end{enumerate}
Deallir bod yr effeithiau hyn yn ychwanegu sŵn at y system ac y dylai ddodi i'r prawf fwyo  afreidrwydd i gywirio'r effeithiau hyn. Deallir pe bai ateb yn cael ei roi am reswm da o leuaf fwy na hanner yr amser, byddai graddio'r sefyllwyr prawf yn dal i gydgyfeirio tuag at ei werth go iawn, er yn arafach. Hyd yn oed mewn gosodiad lle rhoddir mwy na hanner yr atebion am resymau anghywir, ond bod dosbarthiad atebion cywir ac anghywir yn gytbwys, byddai'r model yn dal i allu osgoi dirywiad. Ond beth bynnag, ni ddylai nifer yr eitemau a brofwyd mewn sesiwn prawf gael ei wneud mor fyr ag sy'n bosibl yn ddamcaniaethol, ond cymryd y sŵn hwn i ystyriaeth. Unwaith eto, mae'r system graddio Elo yn gwneud hyn yn ddi-dor gyda'r ``ffwythiant ansicrwydd'' a fynegwyd yn barod. Mae~\cite{pelanek_applications_2016} yn cynnig y ffwythiant ansicrwydd canlynol i ddiweddaru cyfradd diweddariad (sef y gwerth $K$) yn swyddogaeth o nifer yr atebion blaenorol.
\begin{equation}
    (n)=a/(1 + bn)\label{eq:uncertainty-function}
\end{equation}

Lle mae $a$ a $b$ yn gysonion positif ac $n$ yw nifer yr eitemau a atebwyd yn flaenorol. Defnyddir y rhif sy'n deillio fel y gwerth $K$ sy'n lluosi cywiriad graddio ar ôl pob ateb. Unwaith eto, byddwn yn dychwelyd at yr agwedd hon yn y bennod nesaf.

\section{Casgliad}
Cyflwynodd yr adolygiad llenyddiaeth hwn syniadau o sawl maes a cheisiodd eu trefnu yn gyfanwaith cydlynol. O ddadlau seicoieithyddol sy'n cefnogi'r syniad y gellir defnyddio geirfa fel dirprwy ar gyfer cymhwysedd iaith gyffredinol. I gynnig model olrhain gwybodaeth sy'n optimeiddio'r wybodaeth a enillir gan ganlyniadau prawf deuaidd. Yn y bennod nesaf, byddwn yn rhoi'r darnau hyn gyda'i gilydd i adeiladu prawf geirfa adnabod cyflawn a ffwythiannol.