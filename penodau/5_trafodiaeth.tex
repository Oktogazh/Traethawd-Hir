\chapter{Trafodiaeth}
Mae'r bennod olaf hon wedi ei rhannu'n 4 adran. Yn gyntaf, cyfrif o arsylwadau uniongyrchol a allai lywio ymchwilion yn y dyfodol ar y prawf. Yn ail, trafodir cyfyngiadau'r prawf a mater dehongli'r sgoriau. Yn drydydd, esbonir y cyfeiriad ar sut y byddai ymchwil i fod ar y pwnc yn edrych. Yn olaf, cyflwynir casgliad sy'n crynhoi holl gyfraniadau'r draethawd yn ogystal ag ateb gwybodus i'r cwestiwn ymchwil.

\section{Y Prawf ar Arfer}
Mae'r hysbysrwydd yn yr adran hon yn seiledig ar arsylwadau uniongyrchol a wnaeth y awdur wrth ddefnyddio'r prawf mewn amrywiaeth o ieithoedd, neu edrych at rai eraill yn defnyddio'r prawf. Nid ydynt yn cael eu cefnogi gan ddata, ond dim ond mewnwelediadau uniongyrchol, ac felly efallai y byddant yn destun rhagfarn. Fodd bynnag maent yn dod â goleuni i bwyntiau tywyll na ellid eu rhagweld wrth gynllunio'r prawf.

\subsection{Oed a Chysylltiad â Risg}
Wrth edrych at bobl sy'n cymryd y prawf Llydaweg am y tro cyntaf, ymddangosodd bod pobl hŷn yn sgorio'n well na phobl ifanc. Rwy'n cofio'n arbennig ddau berson ifanc a oedd yn ysgolheigion mewn ysgolion dwyieithog tan oedran 18, ac a barhaodd i ddefnyddio'r iaith i ryw raddau wedyn, tra bod o leiaf un person hŷn heb unrhyw addysg ffurfiol yn y Llydaweg, ac na ddarllenodd lyfrau byth yn yr iaith, gan ddysgu'r iaith drwy ryngweithio cymdeithasol achlysurol yn unig. Mae'n bosibl iawn bod gan bobl hŷn lawer o eirfa, ond yr hyn a syrthiodd allan fwyaf oedd pa mor isel oedd sgôr y bobl ifanc. Gan wybod bod y bobl ifanc hyn yn gallu cael sgyrsiau rhugl yn y Llydaweg, disgwylid iddynt wybod y geiriau mwyaf cyffredin, ond roeddent ill dau wedi sgorio o dan 500. Hyd heddiw, gwelwn ddwy esboniad am y duedd hon. Naill ai allai'r eirfa a gafwyd trwy esboniad goddefol i'r iaith yn yr ysgol fod yn isel ei safon. Naill ai mae ymddygiad pobl ifanc wrth wynebu geiriau anhysbys, sef geiriau ffug, yn wahanol. Cymerodd y bobl hŷn a oedd yn cymryd y prawf lawer o amser i feddwl am bob ateb. Cymerodd pobl ifanc y prawf yn gyflym, ac ymddangosodd eu bod yn llai annhueddol rhag beryglu eu hunain, efallai'n barod i dderbyn ystyr lle nad oes un, neu efallai'n anfodlon cydnabod eu anwybodaeth a'u cyfyngiadau. Oherwydd pa mor llym yr oedd y sgôr yn cael ei gostwng wrth gydnabod geiriau anghywir, gallai'r amrywiad annisgwyl hwn mewn perthynas pobl â risg achosi amrywiadau yn y prawf eirfa Llydaweg, waeth beth fo lefel absoliwtedd a lefel eirfa. Dyna pam y cafodd sgoriau geiriau ffug eu ymcychwyn o fewn yr ystod 0–2000 ar gyfer profion yr ieithoedd eraill.

Yn anffodus nid oes mwy iddo na'r sylwadau uniongyrchol hyn. Ond mae'r arsylgiad hwn ar y berthynas â risg yn ymddangos yn gyson â ymchwil mewn gwyddorau cymdeithasol \parencite{wang_does_2023}. Meddylwyd ymgymryd â'r newidyn hwn yn y system sgorio, fel trwy fodelu'r tuedd i gydnabod ysgogwyr a rhyw ffordd o gael i'r sgôr derfynol fapio mwy i lefel absoliwt mewn sgiliau adnabod geirfa, os yw hyn yn bosibl hyd yn oed tra'n cyfrif am strategaethau twyllo. Fodd bynnag, rhoddwyd yr syniad hwn o'r neilltuo am ddau reswm. Yn gyntaf, bwriedir i'r profion fod ar gyfer hunanasesu, i fesur cynnydd unigolion dros amser, nid i gymharu lefel rhwng myfyrwyr. Yn ail, gan fod y prawf wedi'i fwriadu ar gyfer defnydd ailadroddus, disgwylir y bydd y rhai sy'n cymryd y prawf yn addasu eu hymddygiad yn y pen draw i ``gyd-fyw â'u hanwybodaeth'', boed er mwyn optimeiddio eu canlyniadau, sef peidio cydnabod geiriau heb fod yn sicr eu nabod.

\subsection{Gwerth y Prawf fel Offeryn Dysgu}
Fel dysgwr iaith ddechreuwr yn Wcreineg, mae'r awdur yn cael ei amlygu'n rheolaidd i'r iaith mewn gosodiad suddol. Mae'r promt dadansoddi'r canlyniadau prawf wedi profi'n offeryn dysgu defnyddiol iawn. Offeryn sy'n ategu'r esboniad llafar i'r iaith gyda adborth ysgrifenedig wedi'i deilwra, gan feithrin sgiliau cyffredinol. Yn yr ystod isel o raddfeydd mae ychydig o eitemau i ddewis ohonynt o gymharu. Mae hyn yn golygu bod y rhai sy'n cymryd y prawf yn debygol o ddod ar draws ychydig o eitemau hyn mewn unrhyw sesiwn brawf. Yn yr ystod hon, mae'r syniad bod y geiriau go iawn nad ydynt yn cael eu hadnabod mewn sesiwn brawf yn y geiriau nesaf mwyaf defnyddiol i'w dysgu yn gryf. Y nod cychwynnol tu ôl i adeiladu prawf oedd adeiladu brics technolegol a fyddai'n helpu i optimeiddio rhaglenni addysgu yn ddiweddarach. Ond daethpwyd o hyd i allu ateb yn gywir i'r cwestiwn ``Ble i ddechrau nawr''? eisoes yn rhan fawr mewn unrhyw broses addysgu. Mae'r adborth syml sy'n seiliedig ar LLM, er ei fod yn anghwbl, yn ymddangos fel yr agwedd fwyaf defnyddiol ar y prawf hyd yma. Ond wrth gwrs, mae'n dibynnu ar bopeth arall sydd wedi'i drafod hyd yma.

\section{Cyfyngiadau a Dehongliad y Sgôr}
Camddehongliad hawdd i'w wneud am sgôr y prawf fyddai ystyried bod yr holl eiriau islaw'r sgôr derfynol wedi'u meistroli gan y myfyrwyr a'r holl eiriau a werthir uwchlaw'r sgôr derfynol fel rhai nad ydynt yn cael eu nabod ganddynt. Nid yw hyn yn union yr hyn y mae'r canlyniadau'n ei awgrymu. Dylai'r sgôr derfynol gynrychioli'r lefel lle mae myfyriwr yn adnabod dim ond 50\% o'r geiriau, heb ormod o ddealltwriaeth o'u hystyr. Mae gan air a werthir 677 pwynt llai na'r sgôr derfynol siawns o 99\% o gael ei hadnabod yn gywir. Yn groes i hynny, byddai eitem a werthir 677 pwynt yn fwy na sgôr derfynol yn cael ei hadnabod 1\% o'r amser. Gan fod mwy o eiriau yn yr ystodau uchaf, os yw'r sgôr derfynol yn yr ystodau isaf, mae'r 1\% hwn yn ``fwy'' na'r 1\% o ychydig o eitemau mewn ystodau is. Mewn termau syml, mae hyn yn golygu bod, mewn gwirionedd, disgwylir i bron pob gair yn yr ystodau islaw gwerth penodol gael ei wybod yn dda, (gan gynnwys gyda dealltwriaeth gryfach o'u hystyr). Ond gallai llawer o eiriau yn yr ystod uchaf dal gael eu hadnabod.

Mar'r test hwn yn edrych am bwynt torri yng ngwybodaeth y myfyrwyr, yn hytrach na dangos yn union yr holl eiriau y maent yn eu nabod. Gan gymharu eto â'r paradigm CEFR, sydd wedi'i seilio ar ymagwedd ``gallu gwneud'' \parencite{europe_common_2020}, mae'r paradigm hon yn edrych am yr hyn ``na allu ei wneud''. Mewn practis bydd yn gweithio yr un fath ar gyfer mwyafrif o achosion, oherwydd natur normatif mecanwaith diweddaru'r system Elo. Fodd bynnag, mae cyfyngiad posibl i'w aroleuo yma.

Nid yw pobl i'w disgwyl i ddysgu ieithoedd yr un ffordd. Wrth ddefnyddio iaith fel plant, partneriaid, myfyrwyr, ysgolheigion, twristiaid, proffesiynolion neu genhadon, efallai y bydd angen i bobl feistroli meysydd geirfa gwahanol. Mae'n gwestiwn agored i ba raddau y mae geirfa'r dysgwyr gwahanol hyn yn gorgyffwrdd. Ai yw'r amrywiad hwn mewn defnydd yn gwrthdaro â'r dehongliad ystadegol a gyflwynwyd uchod sy'n gwestiwn arall. Yn y pen draw, efallai y bydd yn berthnasol adeiladu profion geirfa clôn er mwyn targedu gwahanol ddemograffeg o fewn yr un iaith. Efallai mai clônio'r prawf yw'r ffordd orau i sicrhau casgliad priodol y ``gallu ei wneud'' o'r ``na allu ei wneud''.

\section{Ymchwil Dyfodol}
Gwelwn dri cyfeiriad gwahanol o ymchwil yn mynd ymlaen. Y cyntaf yw astudio effeithiau silff posibl yn y profion, a'r cyflymder y gellir ystyried bod caledwedd prawf wedi'i gwblhau. Mae'r agwedd hon yn hanfodol i fesur deinameg caffael iaith. Yr ail yw addysg, gan y gellir defnyddio'r prawf eisoes o leiaf i fesur lefelau ``cerrig milltir'' mewn caffael geirfa. Gellid ei ddefnyddio i arbrofi ar wahanol ddulliau addysgu, \textbf{gan gynnwys mewn gosodiadau dwyieithog}. Yn olaf, gellid gwthio'r prawf ymhellach, gan gynnwys mewn amgylcheddau nad ydynt yn WEIRD i astudio amrywiad defnydd iaith ac asesu'r angen am raddfeydd penodol i bwnc.

Gall y ddau gyfeiriad cyntaf yn cael eu ymchwilio'n baralel gyda ychydig o ymdrechion, trwy ganolbwyntio o amgylch dosbarthiadau oedolion o ieithoedd rhanbarthol fel y Gymraeg. Byddai hyn yn troi diffyg adnoddau mewn rhai ieithoedd yn fantais ymchwil. Byddai'r trydydd yn gofyn am fwy o adnoddau ac yn fwyaf tebygol o gydlynu rhyngwladol, gyda gwahanol grwpiau cymdeithasol yn ymwneud ac yn debygol o ganolbwyntio ar IALl.

Yn y maes ehangach o AIED, mae'r dechneg clystyr modulo yn ymddangos yn ffordd gobeithiol o adeiladu profion HAdd yn y dyfodol, y byddai eu heitemau wedi'u cynhyrchu'n un clamp gan LLM. Gellid defnyddio'r dull hwn i galibro eu anhawster cymharol a mesur lefel a chynnydd myfyrwyr mewn meysydd eraill na CAI. Yn enwedig, gallai ymchwil newydd posibl ganolbwyntio ar efelychiadau, i ddod o hyd i beth yw'r paramedrau optimaidd (Sail modwlo, lledaeniad yr eitemau cychwynnol ac ati...) i adeiladu profion o'r fath.

\section{Casgliad}
Wrth ddychwelyd at y cwestiwn gwreiddiol, ni all yr holl elfennau a gasglwyd yn y gwaith hwn ddisprofi'r syniad y gellir creu profion geirfa cyflym, addasol a graddadwy ar gyfer ieithoedd â llai o adnoddau. Yn gyffredinol, dangosodd y dyluniad prawf geirfa hwn nad yw'r ffocws ar ieithoedd â mwy o adnoddau yn astudio caffael iaith yn angheuoldeb. Mae'n dangos bod gweithio gyda llai o ddata yn gallu ein gorfodi i ddod o hyd i atebion gwreiddiol. Mae llawer o heriau wedi'u goresgyn yn y traethawd hwn, er y gallwn ddod o hyd i fwy o gwestiynau newydd nag atebion na'r rhai a ddechreuon ni gyda nhw. Yn ffodus, gofynnir y cwestiynau hyn ynghyd â rhoi dulliau clir ar gyfer dadansoddiad a hypotheseis y gellir ei gwrthbrofi.

Yn ogystal, mae'n ymddangos ei bod yn fesur da i geisio integreiddio ymchwil ddyfodol ar y pwnc yn agosach yn y cyd-destun defnydd o'r ieithoedd hyn. Y cymhelliant cychwynnol i'r gwaith hwn oedd optimeiddio addysg ieithoedd cyfyngedig eu hadnoddau. Ond yn y cyd-destun hwn, a ellir deall y gair optimeiddio o hyd yn y syniad o gael mwy trwy wneud llai? Yn seiliedig ar yr elfennau a gesglir yn y draethawd hwn, mae'n ymddangos bod sgiliau dynol yn tyfu bob amser i gyd-fynd â'u hanghenion defnydd. Yn hyn o beth, efallai mai'r peth sydd ei angen fwyaf i'w optimeiddio yw'r amser a dreulir ar yr ieithoedd hyn. Hynny yw, cael mwy trwy wneud mwy, a chydnabod nad oes rheswm i feddwl bod llwybr byr technolegol yn bodoli ar gyfer problem sydd yn bennaf gymdeithasol. Gobeithio y bydd prawf geirfa cyflym yn dod yn ffordd o roi sylwedd ac ymwybyddiaeth am werth y defnydd hwn. Nid i'w optimeiddio ymhellach, ond i'w fwyhau ac i'w fwynhau.
